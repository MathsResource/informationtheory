\documentclass[]{article}
\usepackage{framed}
\usepackage{amsmath}
\usepackage{amssymb}
\begin{document}



\section{Information}

Information of an event $x_i$ (also known as self-information) is related to the probability of that event by the following formula:
\[  I(x) = -\mbox{log}_2 \left( p(x_i) \right) \]
The less probable an event is, the higher the information of that event. Very unusual events are ``high information" events. Commonly occuring events are ``low information". Consider the table below.
{
\large
\begin{center}
\begin{tabular}{|r|r|}
\hline $p(x_i)$	&	$I(x_i)$	\\ \hline
0.00001	&	16.609640000	\\ \hline
0.0001	&	13.287712000	\\ \hline
0.001	&	9.965784000	\\ \hline
0.01	&	6.643856000	\\ \hline
0.1	&	3.321928000	\\ \hline
0.2	&	2.321928000	\\ \hline
0.4	&	1.321928100	\\ \hline
0.5	&	1.000000000	\\ \hline
0.6	&	0.736965600	\\ \hline
0.7	&	0.514573200	\\ \hline
0.8	&	0.321928090	\\ \hline
0.9	&	0.152003090	\\ \hline
0.95	&	0.074000580	\\ \hline
0.99	&	0.014499570	\\ \hline
0.99999	&	0.000014427	\\ \hline
0.9999999	&	0.000000144	\\ \hline
\end{tabular} 
\end{center}
}

\end{document}

%--------------------------------------------------%
\subsection*{Entropy}

The entropy of a probability distribution is a number that describes the degree of uncertainty or disorder the distribution represents.


\begin{itemize}
\item Assume we have a set of two mutually exclusive propositions (or equivalently, a random experiment with two possible outcomes). Assume all two possibilities are equally likely.

\item Then our advance uncertainty about the eventual outcome is rather small - we know in advance it will be one of exactly two known alternatives.

\item Assume now we have a set of a million alternatives - all of them equally likely - rather than two.

It seems clear that our uncertainty now about the eventual outcome will be much bigger.
\end{itemize}
\textbf{Formal definitions}

Given a discrete probability distribution function f, the entropy H of the distribution (measured in bits) is given by 

\[H=-\sum_{\forall i : f(x_i) \ne 0}^{} f(x_{i}) log_{2} f(x_{i} )\]
\subsection*{Mutual Information}

Theory of the probability of transmission of messages with specified accuracy when the bits of information constituting the messages are subject, with certain probabilities, to transmission failure, distortion, and accidental additions.
%-------------------------------------------------------------------------------%

\subsection{Information Theory}
\begin{itemize}\item 
The fundamental quantities of information theory entropy, relative
entropy, and mutual information are defined as functionals of
probability distributions.
\item In turn, they characterize the behavior
of long sequences of random variables and allow us to estimate the
probabilities of rare events (large deviation theory) and to find
the best error exponent in hypothesis tests.
\item 

The initial questions treated by information theory lay in the
areas of data compression and transmission. The answers are
quantities such as entropy and mutual information, which are
functions of the probability distributions that underlie the
process of communication. 
\item A few definitions will aid the initial
discussion. We repeat these definitions in Chapter 2. The entropy
of a random variable X with a probability mass function p(x) is
defined by $H(X) = -\sum x p(x) log2 p(x)$. 
\end{itemize}


%-------------------------------------------------------------------------------%

We use logarithms to base 2. The entropy will then be measured in
bits. The entropy is a measure of the average uncertainty in the
random variable. It is the number of bits on average required to
describe the random variable.


%-------------------------------------------------------------------------------%

\subsection{Example}Consider a random variable that has a
uniform distribution over 32 outcomes. To identify an outcome, we
need a label that takes on 32 different values. Thus, 5-bit
strings suffice as labels. The entropy of this random variable is
$H(X) = - \sum p(i) log p(i) = log 32 = 5 bits$, (1.2) which
agrees with the number of bits needed to describe X. In this case,
all the outcomes have representations of the same length.





%-----------------------------------------------------------------------------------------------%
{
	\subsection{Information Entrophy}
	Entropy is the uncertainty of a single random variable. We can
	define conditional entropy $H(X|Y)$, which is the entropy of a
	random variable conditional on the knowledge of another random
	variable. The reduction in uncertainty due to another random
	variable is called the mutual information.

	Information entropy is a measure of the uncertainty associated
	with a random variable. The term by itself in this context usually
	refers to the Shannon entropy, which quantifies, in the sense of
	an expected value, the information contained in a message, usually
	in units such as bits.
	
	The Shannon entropy is denoted by $H(X)$ and is defined as
	\begin{equation}
	H(X) =   - \sum_{i=1}^np(x_i)\log_b p(x_i).
	\end{equation}

\subsection{Example}
	
	A source language has 5 symbols A, B, C, D and E.  The associated
	probabilities of these symbols are 0.35, 0.25, 0.20, 0.10 and
	0.10, respectively.
	
	\begin{itemize}
		\item Calculate the entropy of the source language. \item Define a
		Huffman binary code for the source language. \item Calculate the
		efficiency of this code.
	\end{itemize}
}


\section{Information Theory}

Information theory provides a quantitative measure of the information contained in message signal and allows us to determine the capacity of a communication system to transfer this information from source to destination. \\ \bigskip ln this part of the course, we will explore some basic ideas involved in information theory and source coding.



\begin{itemize}
\item Information theory is a process that focuses on the task of quantifying information. 
\item The quantification of information is achieved by identifying viable methods of compressing and communicating data without causing 
and degradation in the integrity of the data. 
\item Information theory can be utilized in a number of different fields, including quantum computing, 
data analysis and cryptography.
\end{itemize}


\begin{itemize}
\item The origin of modern informational theory is usually attributed to Claude E. Shannon.
\item His work A Mathematical Theory of Communication, first published in 1948, 
lays the foundation for the quantification and compression of data into viable units that may be stored for easy retrieval later. 
\item His basic approach provided the tools necessary to enhance the efficiency of early mainframe computer systems, and translated easily into 
the advent of desktop computers during the decade of the 1970’s.


\item As a branch of both electrical engineering and applied mathematics, information theory seeks to uncover the most efficient 
methods of conveying data, within the limits inherent in the data proper. \item The idea is to ensure that the mass transit of data does 
not in any way decrease the quality, even if the data is compressed in some manner. 

\item Ideally, the data can be restored to its original form upon reaching its destination. 
\item In some cases, however, the goal is to allow data in one form to be converted for mass transmission, 
received at the point of termination, and easily converted into a format other than the original without losing any of the transmitted information.
\end{itemize}



\section{What is Information?}

\begin{itemize} \item  What is meant by the ``information" contained in an event?
\item If we are formally to defined a quantitative measure of information contained in an event, this measure should have some intuitive properties such as:
\begin{itemize} \item [1.] Information contained in events ought to be defined in terms of some measure of the uncertainty of the events.
\item [2.] Less certain events ought to contain more information than more certain events.
\item [3.] The information of unrelated / independent events taken as a single event should equal the sum of the information of the unrelated events.
\end{itemize}

\item A natural measure of the uncertainty of an event is the probability of $A$ denoted $P(A)$.
\end{itemize}

\textbf{1) Information sources:}

An information source is an object that produces an event, the outcome of which is selected at
random according to a probability distribution.  \\ \bigskip A practical source in a communication system is a
device that produces messages, and it can be either analog or discrete (we deal mainly
with the discrete sources, since analog sources can be transformed to discrete sources) \\ \bigskip A discrete information source is a
source that has only a finite set of symbols as possible outputs. The set of source symbols is called the
\textbf{\emph{source alphabet}}, and the elements of the set are called \textbf{\emph{ symbols}} or \textbf{\emph{letters}}.

%----------------------------------------------------------------------------------------------------------%

\section{Memory}
\begin{itemize} \item Information sources can be classified as having memory or being memoryless.
\item A source with
memory is one for which a current symbol depends on the previous symbols.\item A memoryless source is
one for which each symbol produced is independent of the previous symbols.

\item A discrete memoryless sources (DMS) can be characterized by the list of the symbols, the
probability assignment to these symbols, and the specification of the rate of generating these symbols by the source.\end{itemize}


%----------------------------------------------------------------------------------------------------------%

\end{document}
